% gepisat_appdx_c.tex
%
% written by Tyler W. Davis
% Imperial College London
%
% 2014-10-29 -- created
% 2014-10-29 -- last updated
%
% ------------
% description:
% ------------
% This TEX file contains Appendix C: Some useful SQL commands for the GePiSaT model documentation.
%
% ----------
% changelog:
% ----------
% 01. modularized chapter [14.10.29]
% 02. newline for each sentence [14.10.29]
% --> simpler for Git version control
%
%% \\\\\\\\\\\\\\\\\\\\\\\\\\\\\\\\\\\\\\\\\\\\\\\\\\\\\\\\\\\\\\\\\\\\\\\\ %%
%% APPENDIX C -- SOME USEFUL SQL COMMANDS
%% //////////////////////////////////////////////////////////////////////// %%
\section{Some Useful SQL Commands}
\label{app:sql}
While working with the GePiSaT database, circumstances may arise that force the user to work directly with the database from the postgreSQL command prompt.  
This appendix highlights some of these circumstances.  
The syntax of each query is reviewed and explained such that it may assist in creating other queries that may be necessary.

%% \\\\\\\\\\\\\\\\\\\\\\\\\\\\\\\\\\\\\\\\\\\\\\\\\\\\\\\\\\\\\\\\\\\\\\\\ %%
%% APPENDIX C.1 -- LIST OF VARIABLES (SQL)
%% //////////////////////////////////////////////////////////////////////// %%
\subsection{List of variables}
\label{app:sqlvar}
\texttt{SELECT DISTINCT varname FROM var\textunderscore list;}\\

\noindent This command queries the field \textit{varname} from the table \texttt{var\textunderscore list} and returns a list of all the variable names currently in the database.  
The \texttt{DISTINCT} keyword limits the query results to only unique values of \textit{varname} (i.e., removes all duplicate results). 

%% \\\\\\\\\\\\\\\\\\\\\\\\\\\\\\\\\\\\\\\\\\\\\\\\\\\\\\\\\\\\\\\\\\\\\\\\ %%
%% APPENDIX C.2 -- FIND MSVIDX (SQL)
%% //////////////////////////////////////////////////////////////////////// %%
\subsection{Find msvidx}
\label{app:sqlmsvidx}
\texttt{SELECT varname, varid, msvidx FROM var\textunderscore list WHERE varname = `VPD' LIMIT 1;}\\

\noindent This query returns the fields \textit{varname}, \textit{varid}, and \textit{msvidx} from the table \texttt{var\textunderscore list} for a specified variable name.  
In this example the specified variable is ``VPD.''  Note that in the query syntax the variable name is placed between a pair of single quotes. 
The variable name can be any of the observation data in the database (see Table \ref{tab:gepisatobs} for observation variables; see also Appendix \ref{app:sqlvar} for querying variable names).  
The \texttt{LIMIT 1} at the end of the query suppresses the number of rows returned by the query to one (since all rows will have the same \textit{varname} and \textit{varid}).  
The station that is associated with the \textit{msvidx} will be chosen by postgreSQL by random.  
The purpose of this command is to associate a variable name with its identifier.

%% \\\\\\\\\\\\\\\\\\\\\\\\\\\\\\\\\\\\\\\\\\\\\\\\\\\\\\\\\\\\\\\\\\\\\\\\ %%
%% APPENDIX C.3 -- DELTE SINGLE VARIABLE DATA (SQL)
%% //////////////////////////////////////////////////////////////////////// %%
\subsection{Delete single variable data}
\label{app:sqldelone}
\texttt{DELETE FROM data\textunderscore set WHERE msvidx LIKE '\%20';}\\

\noindent This query performs the action of deleting data (i.e., rows) from the table \texttt{data\textunderscore set} where the \textit{msvidx} field meets a specific criteria.  
As defined in \S \ref{sec:dbvtdes}, the \textit{msvidx} field is comprised of both the \textit{stationid} and \textit{varid} fields.  
By definition the last two characters in the \textit{msvidx} field are associated with the \textit{varid}.  
This query takes advantage of this knowledge and the knowledge of the \textit{varid} for the variable of interest (see Appendix \ref{app:sqlmsvidx}).  
The \texttt{LIKE} keyword prompts postgreSQL for a ``regular expression'' search string.  
In postgreSQL, the escape character for ``match anything'' is the percent sign (\%).  
In this example, the \textit{varid} is 20, which is associated with the ``VPD'' variable name.  
The `\%20' (note the single quotes) will therefore match all \textit{stationid} parts in the \textit{msvidx} where the \textit{varid} part matches `20.' 

Following the successful query execution, postgreSQL will return the number of fields that were removed from the database.
