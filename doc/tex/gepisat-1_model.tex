% gepisat-1_model.tex
%
% written by Tyler W. Davis
% Imperial College London
%
% 2014-10-29 -- created
% 2014-10-29 -- last updated
%
% ------------
% description:
% ------------
% This TEX file contains Part 1 model overview for the GePiSaT model documentation.
%
% ----------
% changelog:
% ----------
% 01. modularized chapter [14.10.29]
% 02. newline for each sentence [14.10.29]
% --> simpler for Git version control
%
%% \\\\\\\\\\\\\\\\\\\\\\\\\\\\\\\\\\\\\\\\\\\\\\\\\\\\\\\\\\\\\\\\\\\\\\\\ %%
%% PART 1.2 -- MODEL OVERVIEW
%% //////////////////////////////////////////////////////////////////////// %%
\section{Overview of Model Stages}
\label{sec:ovmodel}

%% \\\\\\\\\\\\\\\\\\\\\\\\\\\\\\\\\\\\\\\\\\\\\\\\\\\\\\\\\\\\\\\\\\\\\\\\ %%
%% PART 1.2.1 -- Stage 1
%% //////////////////////////////////////////////////////////////////////// %%
\subsection{Stage 1 - ``Partitioning carbon flux data''}
\label{sec:ovstage1}
The first modeling stage consists of partitioning high time-resolution CO$_{2}$ flux data, $F$, using in-situ photosynthetic photon flux density (PPFD) measurements, $Q$.  
Both $F$ and $Q$ observation pairs are obtainable from networks of eddy covariance flux towers around the world.

%% \\\\\\\\\\\\\\\\\\\\\\\\\\\\\\\\\\\\\\\\\\\\\\\\\\\\\\\\\\\\\\\\\\\\\\\\ %%
%% PART 1.2.2 -- Stage 2
%% //////////////////////////////////////////////////////////////////////// %%
\subsection{Stage 2 - ``Light-use efficiency model''}
\label{sec:ovstage2}
The second modeling stage estimates the light-use efficiency (LUE) base on time-aggregated (e.g., monthly) GPP and time-aggregated, gap-filled PPFD.  
An analysis of the empirical dependencies of LUE on vegetational and environmental factors are investigated in order to yield a simple predictive model of LUE.

%% \\\\\\\\\\\\\\\\\\\\\\\\\\\\\\\\\\\\\\\\\\\\\\\\\\\\\\\\\\\\\\\\\\\\\\\\ %%
%% PART 1.2.3 -- Stage 3
%% //////////////////////////////////////////////////////////////////////// %%
\subsection{Stage 3 - ``Global representation of GPP''}
\label{sec:ovstage3}
The third and final modeling stage generates spatial fields of GPP based on remotely sensed reflectances and predictive LUE (obtained from stage 2).
