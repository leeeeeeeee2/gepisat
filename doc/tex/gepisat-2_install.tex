% gepisat-2_install.tex
%
% written by Tyler W. Davis
% Imperial College London
%
% 2014-10-29 -- created
% 2014-10-29 -- last updated
%
% ------------
% description:
% ------------
% This TEX file contains Part 2 database installation for the GePiSaT model documentation.
%
% ----------
% changelog:
% ----------
% 01. modularized chapter [14.10.29]
% 02. newline for each sentence [14.10.29]
% --> simpler for Git version control
%
%% \\\\\\\\\\\\\\\\\\\\\\\\\\\\\\\\\\\\\\\\\\\\\\\\\\\\\\\\\\\\\\\\\\\\\\\\ %%
%% PART 2.2 -- INSTALLATION
%% //////////////////////////////////////////////////////////////////////// %%
\section{Installation}
\label{sec:dbinstall}
PostgreSQL can be installed on a variety of operating systems. 
A popular alternative to installing PostgreSQL directly onto your computer is to use a virtual computer which operates independently and can be easily ported from one machine to another.  
BitNami\footnotemark \footnotetext{\url{http://bitnami.com}} is a provider of  pre-packaged virtual computers with applications already configured for software development. 
The database for this modeling system was tested on both a native PostgreSQL installation and on the BitNami Linux Application (LAPP) stack. 
The LAPP stack provides a complete PostgreSQL development environment which also includes the PostGIS extensions.  
In order to run a virtual machine, virtualization software is required.  
There are numerous virtualization software providers, however, the most common are VirtualBox\footnotemark \footnotetext{\url{https://www.virtualbox.org}} and VMWare\footnotemark \footnotetext{\url{http://www.vmware.com}}.  

%% \\\\\\\\\\\\\\\\\\\\\\\\\\\\\\\\\\\\\\\\\\\\\\\\\\\\\\\\\\\\\\\\\\\\\\\\ %%
%% PART 2.2.1 -- INSTALLATION USING VIRTUAL MACHINE
%% //////////////////////////////////////////////////////////////////////// %%
\subsection{Virtual machine setup}
\label{sec:dbinstvm}
The following instructions give a step by step overview on how to install the LAPP stack virtual machine on OSX.  

\begin{enumerate}
	\item Download and install VMware Fusion 6\footnotemark \footnotetext{\url{https://www.vmware.com/fr/products/fusion/}}. VMware fusion is not open source and a license purchase is required.
	\item Download Bitnami LAPP Stack\footnotemark \footnotetext{\url{http://bitnami.com/stack/lapp}} (Virtual Machine, LAPP Stack on VMware) and extract the \textbf{.zip} file to your home directory.
	\item Start VMware fusion:
		\begin{enumerate}
		\item Case 1, first time using VMware
			\begin{enumerate}
			\item VMware will offer you to install a virtual machine through several ways. Choose: \\ 
				``More options...'' $\rightarrow$ ``Import an already existing virtual machine''.
			\item Browse to your home directory, where you have extracted the \textbf{.zip} file; the directory should be named:\\ 
				bitnami-lappstack-5.4.24-0-ubuntu-12.04 
			\item Select the \textbf{.ovf} file; it should be named:\\ 
				bitnami-lappstack-5.4.24-0-ubuntu-12.04-VBOX3.ovf
			\item Choose a place to install the virtual machine; the default location on OSX is: ``Documents/Virtual Machines''
			\end{enumerate}
		\item Case 2, add a new virtual machine to an already existing VMware installation
			 \begin{enumerate}
			 \item In VMware, navigate to \texttt{File} $\rightarrow$ \texttt{New...}
			 \item At the prompt for the installation method, follow the same procedure as in Case 1 above.
			 \end{enumerate}
		\end{enumerate}
	\item If you are using a \textbf{wifi connection}:
		\begin{enumerate}
		\item Once the BitNami is installed (you should get an installation report with confirmation of success, partition size, etc.), click on ``Configuration''.
		\item Select ``Network interface controller''.
		\item Tick ``wifi'' and make sure the ``Network interface controller'' is activated. You will be prompted for your \textbf{OSX} admin password.
		\item Close the window and start BitNami.\\
		\textbf{NOTE}: If you forgot to activate the wifi as described above just after the installation process, BitNami configuration will fail on starting up. Run the command: \texttt{sudo poweroff} to power down BitNami.  Go to \texttt{Virtual machine} $\rightarrow$ Configuration... $\rightarrow$ ``Network interface controller'' and repeat the procedure described above. Keyboard shortcut: \textbf{cmd + E}. 
		\end{enumerate}			
	\item Run BitNami
	\item On starting up BitNami, the following will displayed at the top of the screen, including the ip address: \\
		\texttt{*** You can access the application at\\~~~~http://XXX.XXX.X.XX ***} \\
		This address can later be accessed with the \texttt{ifconfig} command.
	\item Sign in with the default username (\texttt{bitnami}) and password (\texttt{bitnami}) at the screen prompts.\\
		You will be prompted for changing the password following the first login.
\end{enumerate}

%% \\\\\\\\\\\\\\\\\\\\\\\\\\\\\\\\\\\\\\\\\\\\\\\\\\\\\\\\\\\\\\\\\\\\\\\\ %%
%% PART 2.2.2 -- INSTALLATION USING NATIVE LINUX
%% //////////////////////////////////////////////////////////////////////// %%
\subsection{Native Linux setup}
\label{sec:dbinstnl}
The following instructions give a basic overview of installing PostgreSQL with PostGIS on a native Linux system.

\begin{enumerate}
    \item Download and install the PostgreSQL core distribution for your 
          operating system\footnotemark 
          \footnotetext{\url{http://www.postgresql.org/download}}
          or use the native application installer (e.g., apt-get) or package 
          manager (e.g., Synaptic Package Manager)
    \item Setup the password for the default postgres user.\\
          \emph{Note:} for BitNami LAPP stack installations, the postgres user 
          already has a default password (e.g., `bitnami') and PostgreSQL can
          be accessed the first time using: \texttt{psql -U postgres -d 
          postgres}
    \begin{enumerate}
        \item Log into PostgreSQL terminal:\\
              $>$~\texttt{sudo -u postgres psql postgres}
        \item Create a new password on the PostgreSQL terminal:\\ 
              $\#$~\textbackslash \texttt{password postgres}
        \item Enter and confirm the new password
        \item Quit PostgreSQL:\\ 
              $\#$~\texttt{\textbackslash q}
    \end{enumerate} 
\end{enumerate}

%% \\\\\\\\\\\\\\\\\\\\\\\\\\\\\\\\\\\\\\\\\\\\\\\\\\\\\\\\\\\\\\\\\\\\\\\\ %%
%% PART 2.2.2 -- RESIZING POSTGRESQL DATABASE
%% //////////////////////////////////////////////////////////////////////// %%
\subsection{Resizing Linux for postgreSQL database}
\label{sec:dbresize}
The database in this model can reach a disk size greater than the default disk size of the BitNami virtual machine (i.e., $\approx$ 17 GB).   
Additionally, some Linux OS disk partitions are not sized adequately for the database installed in the default location (e.g., /usr/lib/postgresql/9.1/main/data).  
Therefore, the database data directory needs to be relocated to a partition with adequate space to hold the observations for the model.  
The following instructions are for moving the default postgreSQL database data directory.

\begin{enumerate}
    \item Create or identify a partition for holding the database data
    \begin{enumerate}
        \item BitNami
        \begin{enumerate}
            \item While the virtual machine is turned off, expand the hard disk 
                  settings to a larger capacity, (e.g. navigate to: VMWare Fusion 
	               $\rightarrow$ Virtual Machine $\rightarrow$ Hard Disk (IDE) 
	               $\rightarrow$ Hard Disk (IDE) Settings $\ldots$ and move the 
	               slider to allocate additional space)
            \item Turn on the virtual machine
            \item List the disk partitions:\\
                  $>$~\texttt{sudo fdisk -l}\\
                  \\
                  The partitioning table will be displayed, providing a breakdown
    	           of device names, start and end positions, number of blocks 
    	           (e.g., size of each device), partition id and associated system
    	           name.  At the top of the list, the total disk size should be 
    	           displayed corresponding to the size selected previously 
    	           (e.g., /dev/sda: 85.9 GB, 167772160 sectors, sector = 512 
    	           bytes).  Note that the blocks listed in the partition table 
    	           correspond the the sectors.  For example, the current table
    	           may only have two devices: /dev/sda1 and /dev/sda2 
    	           corresponding to a Linux and Linux swap system, respectively.
    	           The end position of the second device (i.e., /dev/sda2) should
    	           be less than the number of sectors listed for the drive (e.g., 
    	           end = 33998847).
            \item Determine the main partition of your LAPP stack based on the 
                  results from the previous step (e.g., /dev/sda)
            \item Format the disk, e.g.:\\
                     $>$~\texttt{sudo fdisk /dev/sda}
            \item Create new partition for holding the database by typing the following keystrokes in \texttt{fdisk}:\\
                  \texttt{p}: prints the partition table\\
                  \texttt{n}: creates a new partition\\
                  \texttt{p}: sets new partition type to ``primary''\\
                  \texttt{3}: defines partition number\\
                  \texttt{<default>}: sets the start position to default\\
                  \texttt{<default>}: sets the end position to default\\
                  \\
                  Note: the default start and end positions may fall in a small 
                  gap located between /dev/sda1 and /dev/sda2; if this is 
                  the case, create another primary partition:\\
                  \texttt{n}: creates new partition\\
                  \texttt{p}: sets partition type to ``primary''\\
                  \texttt{4}: defines partition number\\
                  \texttt{<default>}: sets the start position to default\\
                  \texttt{<default>}: sets the end position to default\\
                  \textbf{w}: writes partition table
            \item Restart the virtual machine:\\
                  $>$~\texttt{sudo poweroff}
            \item Make new partition into a file system (Note: in this example
                   /dev/sda4 is the new partition with start position 33998848 
                   and end position 167772159, e.g., 85.9 GB):\\
                   $>$~\texttt{sudo mkfs -t ext3 /dev/sda4}
             \item Create a mounting point for this partition, e.g.:\\
                   $>$~\texttt{sudo mkdir /database}
             \item Mount new partition to a directory.\\
             		Open the fstab document:\\
                   $>$~\texttt{sudo pico /etc/fstab}\\
                   \\
                   Add the following line to the end of the document and save:\\
                   \texttt{/dev/sda4~~~/database~~~ext3~~~defaults~~~0~~~0}
    	\end{enumerate}
        \item Native Linux
        \begin{enumerate}
            \item Create a directory for holding the database on a partition 
                  with adequate space available (e.g., $>$80 GB).  On most Linux 
                  machines, the /home directory is on a large partition. \\
                  $>$~\texttt{mkdir \textasciitilde /Database}
        \end{enumerate}
    \end{enumerate}
    \item Create a data directory in the database directory created in the 
         previous step (e.g., for BitNami):\\
         $>$~\texttt{sudo mkdir /database/data}
   \item Change directory owner to user ``postgres'' (e.g., for BitNami):\\
         $>$~\texttt{sudo chown postgres:postgres /database/data}
   \item Stop postgresql service (e.g., BitNami):\\
         $>$~\texttt{sudo /opt/bitnami/ctrlscript.sh stop}
   \item Initialize new data directory (e.g., BitNami):\\
         $>$~\texttt{su - postgres -c `/opt/bitnami/postgresql/bin/initdb \\
         ~~-D /database/data'}\\
         \\
         Note: a request will be made for postgres user's password, which may
         not be set yet.  If it is not set (i.e., the above command fails for 
         postgres password), first create a password for postgres and then
         try running the command again:\\
         $>$~\texttt{sudo passwd postgres}
   \item Edit postgresql start up script (e.g. BitNami):\\
   		 Backup the \texttt{ctl} script with the following command:\\
         $>$~\texttt{sudo cp /opt/bitnami/postgresql/scripts/ctl.sh \hfill \ldots \\
         ~~/opt/bitnami/postgresql/scripts/ctl.sh.bak}\\
         \\
         Open the \texttt{ctl} script:\\ 
         $>$~\texttt{sudo pico /opt/bitnami/postgresql/scripts/ctl.sh}\\
         \\
         Edit the start-up commands (i.e., the two lines that start with ``POSTGRESQL\textunderscore START'' and ``POSTGRESQL\textunderscore STOP'') by changing the directory path following the ``-D'' with the new database data data directory (e.g., ``/database/data'')
   \item Start postgresql service (e.g., BitNami):\\
         $>$~\texttt{sudo /opt/bitnami/ctrlscript.sh start}
   \item Check that postgresql is running correctly:\\
         $>$~\texttt{ps auxw | grep postgresql | grep -- -D}\\
         \\
         Note the directory listed following the -D.  If it is the right one 
         (e.g., /database/data), proceed.
   \item Create postgres password:\\
         Log into database:\\ 
         $>$~\texttt{psql -U postgres -d postgres}\\
         \\
         Set password:\\ 
         $\#$~\textbackslash \texttt{password postgres} \\
         \\
         Exit psql:\\ 
         $\#$~ \textbackslash \texttt{q}
   \item Stop postgresql service (e.g. BitNami):\\
         $>$~\texttt{sudo /opt/bitnami/ctrlscript.sh stop}
   \item Edit pg\textunderscore hba.conf file in the newly established data directory (e.g. 
         BitNami):\\
         $>$~\texttt{su - postgres -c `pico /database/data/pg\textunderscore hba.conf'}\\
         \\
         Change all the instances of ``trust'' to ``md5''
   \item Follow setup instructions (\S \ref{sec:dbsetup}) to create the 
         database
\end{enumerate}

Note that in the old database can still be accessed by stopping the postgresql service, replacing the start-up script (e.g., the ``ctl.sh'' script located in /opt/bitnami/postgresql/scripts directory) with the original (i.e., ``ctl.sh.bak'') and restarting the service.
