% gepisat-3_enviro.tex
%
% written by Tyler W. Davis
% Imperial College London
%
% 2014-10-29 -- created
% 2015-03-18 -- last updated
%
% ------------
% description:
% ------------
% This TEX file contains Part 3 programming environment for the GePiSaT model documentation.
%
% ----------
% changelog:
% ----------
% 01. modularized chapter [14.10.29]
% 02. newline for each sentence [14.10.29]
% --> simpler for Git version control
% 03. noted pip install of psycopg2 in Canopy [15.03.18]
%
%% \\\\\\\\\\\\\\\\\\\\\\\\\\\\\\\\\\\\\\\\\\\\\\\\\\\\\\\\\\\\\\\\\\\\\\\\ %%
%% PART 3.1 -- PROGRAMMING ENVIRONMENT
%% //////////////////////////////////////////////////////////////////////// %%
\section{Environment Setup}
\label{sec:mes}
The core of the modeling is performed using open-source programming languages (e.g., Python and R). 
In order to facilitate the model, specific configurations were made to the working environments which these programs operate under.  
These configurations often accompany having specific library files installed within the working environment that the model is executed under.  
This section overviews the necessary working environment configurations to operate the model code described in this reference.

%% \\\\\\\\\\\\\\\\\\\\\\\\\\\\\\\\\\\\\\\\\\\\\\\\\\\\\\\\\\\\\\\\\\\\\\\\ %%
%% PART 3.1.1 -- PROGRAMMING ENVIRONMENT: PYTHON
%% //////////////////////////////////////////////////////////////////////// %%
\subsection{Python environment}
\label{sec:mespy}
The most efficient method of quickly gaining access to all the necessary Python libraries and modules (except psycopg2) is to download an interactive development environment (IDE).  
Two notable Python IDEs are the Enthought Python Distribution\footnotemark \footnotetext{https://www.enthought.com/products/canopy/} (now called ``Canopy'') and Google spyder$^{2}$\footnotemark \footnotetext{http://code.google.com/p/spyderlib/} (previously called ``Pydee'').  
Both Canopy and spyder provide a comprehensive Python developing and programming environment for scientific analysis and visualization.  
There is a free version of Canopy as well as a free one-year academic license for registered users with a school or university e-mail address. 
Spyder is open-source software available for Windows, Mac OSX, and Linux operating systems; however it has not been used or tested.  
This text assumes the use of the Enthought Canopy IDE.
Table \ref{tab:pylib} presents a list of Python libraries that are used in this model and the model files in which they are used.

To overcome the lack of native compatibility between Canopy and the psycopg2 module, psycopg2 can be installed as ``second party'' software by means of pip (Python package manager).
In a native Linux system and using the Canopy Python environment as the default Python environment, run: \texttt{pip install psycopg2} on the command line.
Note that for use of Canopy IDE outside a virtual machine where the GePiSaT database is installed may not be possible (although untried as of this writing).
The main advantage of the Canopy IDE is the easy of installing the HDF4 libraries (for reading the MODIS EVI data), which is rather complicated to successfully achieve (in the writer's opinion).  
For all other circumstances, Canopy may be used separately for processing HDF4 data (via table\textunderscore maker.py), whilst all other Python programs may be run from a separate Python environment.  

%% ------------------------------------------------------------------------ %%
%% tab:pylib | List of Python libraries
%% ------------------------------------------------------------------------ %%
\begin{table}[h]
    \caption{List of Python libraries used in this model.}
    \label{tab:pylib}
    \begin{tabular}{l p{6cm} c c c}
        \hline
        \textbf{Library} & \textbf{Description} & \textbf{[1]} & \textbf{[2]} & \textbf{[3]}\\
        \hline
        datetime & Timestamp handling & x & ~ & x \\
        
        glob & File searching & ~ & x & x \\
        
        numpy & Numerical arrays and operators & x & ~ & x\\
        
        os.path & Directory and file handling & x & x & x\\
        
        psycopg2 & PostgreSQL database management & x & x & ~\\
        
        pyhdf & HDF file access (SD) & ~ & ~ & x\\
        
        re & Regular expression support & ~ & ~ & x\\
        
        scipy.io & netCDF file access & ~ & ~ & x \\
        
        scipy.optimize & Non-linear least squares (curve fit) & x & ~ & 
        ~ \\
        
        scipy.special & Error function (erf) & x & ~ & ~ \\
        
		scipy.stats & Student's t-test & x & ~ & ~ \\

        sys & Model run termination & x & x &  ~ \\
        \hline
        \multicolumn{5}{l}{\footnotesize Included in [1] model.py, [2] db\_setup.py, or [3] table\_maker.py}
    \end{tabular}
\end{table}

%% \\\\\\\\\\\\\\\\\\\\\\\\\\\\\\\\\\\\\\\\\\\\\\\\\\\\\\\\\\\\\\\\\\\\\\\\ %%
%% PART 3.1.2 -- PROGRAMMING ENVIRONMENT: R
%% //////////////////////////////////////////////////////////////////////// %%
\subsection{R environment}
\label{sec:mesre}
Similar to the Python, the R programming language benefits from having an open source IDE for programming and development.  
One such IDE is RStudio\footnotemark. \footnotetext{http://www.rstudio.com/ide/}
