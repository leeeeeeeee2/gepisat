% gepisat-1_intro.tex
%
% written by Tyler W. Davis
% Imperial College London
%
% 2014-10-29 -- created
% 2014-10-29 -- last updated
%
% ------------
% description:
% ------------
% This TEX file contains Part 1 introduction for the GePiSaT model documentation.
%
% ----------
% changelog:
% ----------
% 01. modularized chapter [14.10.29]
% 02. newline for each sentence [14.10.29]
% --> simpler for Git version control
%
%% \\\\\\\\\\\\\\\\\\\\\\\\\\\\\\\\\\\\\\\\\\\\\\\\\\\\\\\\\\\\\\\\\\\\\\\\ %%
%% PART 1.1 -- INTRODUCTION
%% //////////////////////////////////////////////////////////////////////// %%
\section{Introduction}
\label{sec:intro}
This project is aimed to develop a modeling system for global hindcasting and analysis of spatial and temporal patterns in terrestrial gross primary production (GPP).  
This system takes a simplistic approach which makes the best use of observational data (from flux towers, meteorological stations, and remote-sensing satellites) while defensibly representing the principal eco-physiological processes that govern GPP.  
The modeling system is divided into three stages.  
The basis of the system is an efficient database structure designed to hold the variety of observational data necessary to complete each stage of the modeling.  
This modeling work strives for clarity and uniformity so that it may be used by researchers across disciplines.  
The use of open-source software (i.e., PostgreSQL) and programming languages (e.g., Python and R) allows for portability and transparency.  
The model will invite a range of applications to the analysis of climate and CO$_{2}$ change impacts on ecosystem processes.
