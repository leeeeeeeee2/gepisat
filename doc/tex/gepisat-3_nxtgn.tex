% gepisat-3_nxtgn.tex
%
% written by Tyler W. Davis
% Imperial College London
%
% 2015-03-18 -- created
% 2015-03-20 -- last updated
%
% ------------
% description:
% ------------
% This TEX file contains the next-generation light-use efficiency model.
%
% ----------
% changelog:
% ----------
% 01. modularized chapter [15.03.18]
% 02. newline for each sentence [15.03.18]
% --> simpler for Git version control
% 03. added subsections [15.03.20]
%
%% \\\\\\\\\\\\\\\\\\\\\\\\\\\\\\\\\\\\\\\\\\\\\\\\\\\\\\\\\\\\\\\\\\\\\\\\ %%
%% PART 3.2 -- NEXT-GENERATION MODEL DEVELOPMENT
%% //////////////////////////////////////////////////////////////////////// %%
\section{Next-Generation Model Development}
\label{sec:nxtgen}
At the heart of GePiSaT is a production efficiency or ``diagnostic'' model for estimating monthly GPP. 
Similar to most other production efficiency models, GPP is expressed as a function of absorbed light, $I_{abs}$.
The basic equation for monthly GPP takes the form:
%% ------------------------------------------------------------------------ %%
%% eq:pem | Production efficiency model
%% ------------------------------------------------------------------------ %%
\begin{equation}
\label{eq:pem}
    \text{GPP} = \varepsilon\; I_{abs} 
\end{equation}

\noindent where:\\
\indent GPP = gross primary production [mol~C~m$^{-2}$~mo$^{-1}$];\\
\indent $I_{abs}$ = absorbed photosynthetic light; [mol~photons~m$^{-2}$~mo$^{-1}$];\\
\indent $\varepsilon$ = light-use efficiency [mol~C~mol~photons$^{-1}$].\\

\noindent The light-use efficiency, $\varepsilon$, is therefore defined as the ratio of GPP to absorbed light. 

The absorbed photosynthetic light is defined as the fraction of absorbed photosynthetically active radiation (PAR) and is calculated as the product of fPAR and PPFD (i.e., PAR in units of quanta).

Despite the simplicity of Eq. \ref{eq:pem}, this basic model performs rather well at a few of the flux tower sites. 
However, in the effort of building a universal model of terrestrial GPP, Eq. \ref{eq:pem} does not capture the global dynamics of GPP; therefore, a next-generation model is proposed based on \emph{The Coordination Hypothesis}, which states \parencite{maire12}:
\begin{quote}
	``photosynthesis is co-limited by the carboxylation and regeneration of ribulose-1,5-bisphosphate (RuBP)''
\end{quote}
where $J_\text{max}$ limitation (i.e., the light harvesting capacity) of photosynthesis is accounted by the hypothesis of an optimality between the cost and benefit of maintaining a certain light-harvesting capacity.
The theoretical model, using the empirical expression for the potential rate of electron transport by \cite{smith37}, takes the following form:
%% ------------------------------------------------------------------------ %%
%% eq:nglue | Production efficiency model
%% ------------------------------------------------------------------------ %%
\begin{equation}
\label{eq:nglue}
    \text{GPP} = \phi_{\circ}\; \text{f}_\alpha\; I_{abs}\; \sqrt{m^2 - c^{2/3}\; m^{4/3}}
\end{equation}

\noindent where:\\
\indent $\phi_{\circ}$ = intrinsic quantum efficiency [0.093~mol~C~mol~photons$^{-1}$];\\
\indent $c$ = maintenance cost of light-harvesting capacity [0.41];\\
\indent $I_{abs}$ = absorbed photosynthetic light [mol~photons~m$^{-2}$~mo$^{-1}$];\\
\indent f$_\alpha$ = index of plant available moisture;\\
\indent $m$ = substrate limitation term.\\

Equation \ref{eq:nglue} introduces two constants (i.e., $\phi_{\circ}$ and $c$) and two variables (i.e., f$_\alpha$ and $m$).
The intrinsic quantum efficiency, \nomenclature{$\phi_{\circ}$}{intrinsic quantum efficiency [0.093~mol~C~mol~photons$^{-1}$]}$\phi_{\circ}$, is assumed a constant value of 0.093 mol~C~mol~photons$^{-1}$, based on the mean of eleven measured species by \cite{long93}. 
The cost for a plant to have a certain light-harvesting capacity, \nomenclature{$c$}{maintenance cost of light-harvesting capacity [0.41]}$c$, set to a constant value 0.41, is empirically derived from reanalysis data of $J_\text{max}$ to $V_\text{cmax}$ ratios presented by \cite{kattge07}.

The index of plant available moisture, f$_\alpha$, is based on the Cramer-Prentice bioclimatic moisture index, $\alpha^\star$ (see \S \ref{sec:gepcp}). 
To maintain a coefficient that varies between 0 and 1 and to emphasize the effect of low soil moisture on limiting GPP, f$_\alpha$ is set equal to the quarter power of $\alpha^\star$ divided by its maximum value or simply: 
\nomenclature{$\text{f}_\alpha$}{index of plant available moisture}
$\text{f}_\alpha = \left(\alpha^\star/1.26\right)^{1/4}$.

The substrate limitation term, $m$, is given by:
%% ------------------------------------------------------------------------ %%
%% eq:msimp | The simplification of m
%% ------------------------------------------------------------------------ %%
\nomenclature{$m$}{substrate limitation term [unitless]}%
\begin{equation}
\label{eq:m}
    m = \frac{c_a - \Gamma^\star}
             {c_a 
              + 2 \Gamma^\star 
              + 3 \Gamma^\star \sqrt{
                1.6\; \eta^\star\; D\; \left[
                  \beta^\star\: \left(K + \Gamma^\star\right)
                \right]^{-1}
              }
             }
\end{equation}

\noindent where:\\
\indent $\beta^\star$ = standardized cost ratio [244.0];\\
\indent $\eta^\star$ = relative viscosity of water [unitless];\\
\indent $c_a$ = partial pressure of atmospheric CO$_2$ [Pa];\\
\indent $D$ = vapor pressure deficit [Pa];\\
\indent $\Gamma^\star$ = photorespiratory compensation point [Pa];\\
\indent $K$ = Michaelis-Menten rubisco-limited photosynthesis coefficient [Pa].\\

\noindent The following subsections discuss each of the variables used in the calculation of $m$.

%% \\\\\\\\\\\\\\\\\\\\\\\\\\\\\\\\\\\\\\\\\\\\\\\\\\\\\\\\\\\\\\\\\\\\\\\\ %%
%% PART 3.2.1 -- THE STANDARDIZED COST RATIO
%% //////////////////////////////////////////////////////////////////////// %%
\subsection{The Standardized Cost Ratio}
\label{sec:beta}
\nomenclature{$\beta^\star$}{standardized cost ratio [244.0]}%
The semi-empirical constant, $\beta^\star$, reflects the ratio of the cost factor for carboxylation and the cost factor for transpiration at standard temperature (i.e., 25 ${}^\circ$C) at sea level. The derivation of $\beta^\star$ is based on \emph{The Least-Cost Hypothesis}, which states \parencite{prentice14}:
\begin{quote}
	''plant leaves minimize the summed unit costs of transpiration and carboxylation.''
\end{quote}
To begin, we introduce $\chi$ as the ratio between leaf internal, $c_i$, and ambient, $c_a$, mole fractions of CO$_2$, which is known to decline with vapor pressure deficit, $D$, in the following form:
%% ---------------------------------------------------------------%%
%% eq:chi | Chi
%% ---------------------------------------------------------------%%
\begin{equation}
\nomenclature{$\chi$}{ratio of leaf internal to ambient mole fraction of CO$_2$ [unitless]}%
\label{eq:chi}
    \chi = \frac{\xi}{\xi + \sqrt{D}}
\end{equation}

\noindent where $\xi$ is carbon cost of water, which may be simply expressed by:
%% ---------------------------------------------------------------%%
%% eq:xi | Xi
%% ---------------------------------------------------------------%%
\nomenclature{$\xi$}{carbon cost of water [unitless]}%
\begin{equation}
\label{eq:xi}
    \xi = \sqrt{\frac{K\; b}{1.6\; a}} = \sqrt{\frac{K\; \beta^\star}{1.6\; \eta^\star}}
\end{equation}

\noindent where:\\
\indent $a$ = carbon cost of maintaining transpiration for assimilation; \\
\indent $b$ = cost of maintaining photosynthetic proteins for assimilation;\\
\indent $K$ = Michaelis-Menten rubisco-limited photosynthesis coefficient [Pa];\\
\indent $\eta^\star$ = relative viscosity of water.\\

Based on Eq. \ref{eq:xi}, $\beta^\star$ is equal to $\eta^\star\: b/a$.
Taking the inverse logistic sigmoidal function of $\chi$, substituting the definition of $\xi$ from Eq. \ref{eq:xi} into Eq. \ref{eq:chi}, the temperature and elevation effects of $D$, $K$, and $\eta^\star$ on $\chi$ may be explicitly expressed (via partial differentiation):
%% ---------------------------------------------------------------%%
%% eq:whe | Wang Han equation
%% ---------------------------------------------------------------%%
\begin{equation}
\label{eq:whe}
    \ln \left(\frac{\chi_o}{1 - \chi_o}\right) = 4.64 + 
    	0.0545\: \left(T_{air} - 25\right) - 
    	0.5\; \ln (D) - 
    	8.15\times 10^{-5}\; z
\end{equation}

\noindent where:\\
\indent $T_{air}$ = air temperature [${}^\circ$C];\\
\indent $D$ = vapor pressure deficit [Pa];\\
\indent $\chi_o$ = optimal value of $\chi$ [unitless];\\
\indent $z$ = altitude above mean sea level [m].\\

\noindent At mean sea level (i.e., $z = 0$), standard temperature (i.e., $T_{air} = 25\;{}^\circ$C), and an ideal vapor pressure deficit (i.e., $D = 1000$ Pa), the optimal value of $\chi$ is 0.767. Substituting this value of $\chi$ and the expression of $\xi$, given in Eq. \ref{eq:xi} (where $K = 70.84$ Pa at 25 ${}^\circ$C), into Eq. \ref{eq:chi}, $\beta^\star$ is found to be 244.0. 

%% \\\\\\\\\\\\\\\\\\\\\\\\\\\\\\\\\\\\\\\\\\\\\\\\\\\\\\\\\\\\\\\\\\\\\\\\ %%
%% PART 3.2.2 -- RELATIVE VISCOSITY OF WATER
%% //////////////////////////////////////////////////////////////////////// %%
\subsection{Relative Viscosity of Water}
\label{sec:eta}
The viscosity of water at a given temperature and pressure, $\eta$, may be calculated based on the methodology of \cite{huber09}. 
The relative viscosity of water with respect to its value at 25 ${}^\circ$C, $\eta^\star$, is calculated by the following ratio:
%% ---------------------------------------------------------------%%
%% eq:ns | Relative viscosity of water
%% ---------------------------------------------------------------%%
\nomenclature{$\eta^\star$}{relative viscosity of water [unitless]}%
\begin{equation}
\label{eq:ns}
    \eta^\star = \frac{\eta}{\eta_{25}}
\end{equation}

\noindent where:\\
\indent $\eta$ = viscosity of water at a given temperature and pressure [Pa s];\\
\indent $\eta_{25}$ = viscosity at 25 ${}^\circ$C and standard pressure [8.90$\times 10^-4$ Pa s].\\

%% \\\\\\\\\\\\\\\\\\\\\\\\\\\\\\\\\\\\\\\\\\\\\\\\\\\\\\\\\\\\\\\\\\\\\\\\ %%
%% PART 3.2.3 -- AMBIENT PARTIAL PRESSURE OF CO2
%% //////////////////////////////////////////////////////////////////////// %%
\subsection{Ambient Partial Pressure of CO$_2$}
\label{sec:ca}
The partial pressure of CO$_2$, $c_a$, is calculated based on the observed annual atmospheric CO$_2$ concentration (see \S \ref{sec:gepnoaa}). 
The conversion can be done knowing the total atmospheric pressure, $P_{atm}$, and using Dalton's Law of Partial Pressure:
%% ---------------------------------------------------------------%%
%% eq:pp | Partial Pressure Convertion
%% ---------------------------------------------------------------%%
\begin{equation}
\label{eq:pp}
    p_x = 1\times 10^{-6}\; ppm_x\; P_{atm}
\end{equation}

\noindent where:\\
\indent $p_x$ = partial pressure of gas \textit{x} [Pa];\\
\indent $ppm_x$ = parts-per-million concentration of gas \textit{x} [ppm];\\
\indent $P_{atm}$ = total atmospheric pressure [Pa].\\

\noindent For instances when $P_{atm}$ is unknown, the elevation-dependent atmospheric pressure may be computed using the Barometric Formula \parencite{allen98}:
%% ---------------------------------------------------------------%%
%% eq:pz | Atmospheric pressure as a function of elevation
%% ---------------------------------------------------------------%%
\nomenclature{$P_{atm}$}{total atmospheric pressure [Pa]}%
\begin{equation}
\label{eq:pz}
    P_{atm}\left( z \right) = P_{\circ} \left( 
    	1 - \frac{L\; z}{T_{\circ}} 
    \right)^{\frac{g\; M_a}{R_u\; L}}
\end{equation}

\noindent where:\\
\indent $z$ = altitude above mean sea level [m];\\
\indent $g$ = standard gravity [9.80665 m s$^{-2}$];\\
\indent $P_{\circ}$ = base atmospheric pressure [101325 Pa];\\
\indent $L$ = temperature lapse rate [0.0065 K m$^{-2}$];\\
\indent $T_{\circ}$ = base temperature [298.15 K];\\
\indent $M_a$ = molecular weight for dry air [0.028963 kg mol$^{-1}$];\\
\indent $R_u$ = universal gas constant [8.3145 J mol$^{-1}$ K$^{-1}$].\\

%% \\\\\\\\\\\\\\\\\\\\\\\\\\\\\\\\\\\\\\\\\\\\\\\\\\\\\\\\\\\\\\\\\\\\\\\\ %%
%% PART 3.2.4 -- VAPOR PRESSURE DEFICIT
%% //////////////////////////////////////////////////////////////////////// %%
\subsection{Vapor Pressure Deficit}
\label{sec:d}
The vapor pressure deficit, $D$, is based on the unit conversion of the observed VPD (see \S \ref{sec:gepvpd}) from kPa to Pa.

%% \\\\\\\\\\\\\\\\\\\\\\\\\\\\\\\\\\\\\\\\\\\\\\\\\\\\\\\\\\\\\\\\\\\\\\\\ %%
%% PART 3.2.5 -- PHOTORESPIRATORY COMPENSATION POINT
%% //////////////////////////////////////////////////////////////////////// %%
\subsection{Photorespiratory Compensation Point}
\label{sec:gs}
\nomenclature{$\Gamma^\star$}{photorespiratory compensation point [Pa]}%
The photorespiratory compensation point, $\Gamma^\star$, depends on both the atmospheric pressure and temperature.
\cite{bernacchi01} provides the definitive equation for $\Gamma^\star$:
%% ---------------------------------------------------------------%%
%% eq:gsbasic | Basic equation of Gamma*
%% ---------------------------------------------------------------%%
\begin{equation}
\label{eq:gsbasic}
    \Gamma^\star = \frac{O\: K_c\: V_\text{o,max}}
                        {2\: K_o\: V_\text{c,max}}
\end{equation}

\noindent where:\\
\indent $O$ = partial pressure of atmospheric O$_2$, [Pa];\\
\indent $K_c$ = Michaelis-Menten constant for carboxylation [Pa];\\
\indent $K_o$ = Michaelis-Menten constant for oxygenation [Pa];\\
\indent $V_\text{c,max}$ = maximum rate of carboxylation [$\mu$mol~m$^{-2}$~s$^{-1}$];\\
\indent $V_\text{o,max}$ = maximum rate of oxygenation [$\mu$mol~m$^{-2}$~s$^{-1}$].\\

\noindent The temperature-dependency of $\Gamma^\star$ is derived from the temperature dependencies of $K_c$, $K_o$, $V_\text{c,max}$, and $V_\text{o,max}$, which have been empirically solved by \cite{bernacchi01}. 
Substituting the temperature-dependency equations for each term in Eq. \ref{eq:gsbasic}, results in the following expression:
%% ---------------------------------------------------------------%%
%% eq:gst | Temperature dependency of Gamma*
%% ---------------------------------------------------------------%%
\begin{equation}
\label{eq:gst}
    \Gamma^\star = \exp \left(7.472 - \frac{\Delta H_a}{R_u\: T_k}\right) \frac{O}{2}
\end{equation}

\noindent where:\\
\indent $\Delta H_a$ = activation energy for $\Gamma^\star$ [37$\,$830 J mol$^{-1}$];\\
\indent $R_u$ = universal gas constant [8.3145 J mol$^{-1}$ K$^{-1}$];\\
\indent $O$ = partial pressure of atmospheric O$_2$, [Pa];\\
\indent $T_k$ = leaf temperature [K].\\

\noindent To account for changes in the atmospheric pressure, the partial pressure of O$_2$ may be expressed in terms of molar fraction based on Dalton's Law (see Eq. \ref{eq:pp}):
%% ---------------------------------------------------------------%%
%% eq:gstpa | Temperature & pressure dependency of Gamma*
%% ---------------------------------------------------------------%%
\begin{equation}
\label{eq:gstpa}
    \Gamma^\star = 5\times 10^{-7}\: ppm_{O_2}\: P_{atm}\: \exp \left(7.472 - \frac{\Delta H_a}{R_u\: T_k}\right)
\end{equation}

\noindent where $ppm_{O_2}$ is the molar fraction of atmospheric O$_2$. Assuming a constant value for $ppm_{O_2}$ (i.e., 209476 ppm), $\Gamma^\star$ may be simplified to:
%% ---------------------------------------------------------------%%
%% eq:gstpb | Temperature & pressure dependency of Gamma*
%% ---------------------------------------------------------------%%
\begin{equation}
\label{eq:gstpb}
    \Gamma^\star = \exp \left(5.216 - \frac{\Delta H_a}{R_u\: T_k}\right)\: P_{atm}
\end{equation}

\noindent where:\\
\indent $\Delta H_a$ = activation energy for $\Gamma^\star$ [37$\,$830 J mol$^{-1}$];\\
\indent $R_u$ = universal gas constant [8.3145 J mol$^{-1}$ K$^{-1}$];\\
\indent $P_{atm}$ = atmospheric pressure [Pa];\\
\indent $T_k$ = leaf temperature [K].\\

\noindent Due to a general lack of understanding, the leaf temperature in Eq. \ref{eq:gst}, $T_k$, may be substituted by the ambient air temperature, $T_{air}$, in units of Kelvin. 

%% \\\\\\\\\\\\\\\\\\\\\\\\\\\\\\\\\\\\\\\\\\\\\\\\\\\\\\\\\\\\\\\\\\\\\\\\ %%
%% PART 3.2.6 -- MICHAELIS-MENTON COEFFICIENT
%% //////////////////////////////////////////////////////////////////////// %%
\subsection{Michaelis-Menten Coefficient of Photosynthesis}
\label{sec:k}
\nomenclature{$K$}{Michaelis-Menten coefficient of photosynthesis [Pa]}%
The Michaelis-Menten coefficient of Rubisco-limited photosynthesis is a function of the Rubisco photosynthetic rates for O$_2$ and CO$_2$ \parencite{farquhar80}:
%% ------------------------------------------------------------------------ %%
%% eq:michaelis | Michaelis Menten coefficient
%% ------------------------------------------------------------------------ %%
\begin{equation}
\label{eq:michaelis}
	K = K_c\: \left( 1 + \frac{O}{K_o} \right)
\end{equation}

\noindent where:\\
\indent $K_c$ = Michaelis-Menten constant for CO$_2$ [Pa];\\
\indent $K_o$ = Michaelis-Menten constant for O$_2$ [Pa];\\
\indent $O$ = partial pressure of atmospheric O$_2$, [Pa].\\

\noindent The Michaelis-Menton constants for CO$_2$ and O$_2$ ($K_c$ and $K_o$, respectively) have temperature dependencies, which have been empirically fit to the Arrhenius function and normalized to 25~$^\circ$C \parencite{farquhar80}:
%% ------------------------------------------------------------------------ %%
%% eq:kcko | Michaelis Menten Kc & Ko coefficients
%% ------------------------------------------------------------------------ %%
\nomenclature{$K_c$}{Michaelis-Menten constant for CO$_2$ [Pa]}%
\nomenclature{$K_o$}{Michaelis-Menten constant for O$_2$ [Pa]}%
\begin{subequations}
\label{eq:kcko}
\begin{align}
	K_c&=K_{c25}\: \exp \left[ 
    	\frac{\Delta H_{a,c}\: \left(T_k-298.15\right)}{298.15\: R_{u}\: T_k}
    \right] \label{eq:kc} \\
    K_o&=K_{o25}\: \exp \left[ 
    	\frac{\Delta H_{a,o}\: \left(T_k-298.15\right)}{298.15\: R_{u}\: T_k}
    \right] \label{eq:ko}
\end{align}
\end{subequations}

\noindent where:\\
\indent $K_{c25}$ = Michaelis-Menten constant for CO$_2$ at 25~$^{\circ}$C [Pa];\\
\indent $K_{o25}$ = Michaelis-Menten constant for O$_2$ at 25~$^{\circ}$C [Pa];\\
\indent $\Delta H_{a,c}$ = activation energy for carboxylation [79$\,$430 J mol$^{-1}$];\\
\indent $\Delta H_{a,o}$ = activation energy for oxygenation [36$\,$380 J mol$^{-1}$];\\
\indent $R_{u}$ = universal gas constant [8.3145 J mol$^{-1}$ K$^{-1}$];\\
\indent $T_k$ = leaf temperature [K].\\

\noindent Once again, leaf temperature, as in Eqns. \ref{eq:michaelis} and \ref{eq:kcko}, may be substituted by the ambient air temperature, $T_{air}$, converted to units of Kelvin. 

The partial pressure values of $K_{c25}$ and $K_{o25}$ are based on the empirical temperature dependencies given by \cite{bernacchi01}, in mole fractions, converted to partial pressures by Dalton's Law (see Eq. \ref{eq:pp}):
%% ------------------------------------------------------------------------ %%
%% eq:kcko25 | Michaelis Menten Kc & Ko coefficients
%% ------------------------------------------------------------------------ %%
\nomenclature{$K_{c25}$}{Michaelis-Menten coefficient of carboxylation at 25~$^\circ$C [Pa]}%
\nomenclature{$K_{o25}$}{Michaelis-Menten coefficient of oxygenation at 25~$^\circ$C [Pa]}%
\begin{subequations}
\label{eq:kcko25}
\begin{align}
	K_{c25}&=1\times 10^{-6} \exp \left[ 38.05 - 
    	\frac{\Delta H_{a,c}}{298.15\: R_{u}}
    \right]\: P_{atm} \label{eq:kc25} \\
    K_{o25}&=1\times 10^{-3} \exp \left[ 20.30 - 
    	\frac{\Delta H_{a,o}}{298.15\: R_{u}}
    \right]\: P_{atm} \label{eq:ko25}
\end{align}
\end{subequations}

\noindent Assuming that the experiments conducted at the University of Illinois at Urbana-Champaign were performed under atmospheric pressure (at elevation 227 m above mean sea level), the partial pressure of $K_{c25}$ is 39.97 Pa and the partial pressure of $K_{o25}$ is 27$\,$480 Pa and are invariant to changes in atmospheric pressure.

% @TODO: write K as a function of Patm and Tk.