% gepisat-2_setup.tex
%
% written by Tyler W. Davis
% Imperial College London
%
% 2014-10-29 -- created
% 2015-03-18 -- last updated
%
% ------------
% description:
% ------------
% This TEX file contains Part 2 database setup for the GePiSaT model documentation.
%
% ----------
% changelog:
% ----------
% 01. modularized chapter [14.10.29]
% 02. newline for each sentence [14.10.29]
% --> simpler for Git version control
% 03. minor text update [15.03.18]
%
%% \\\\\\\\\\\\\\\\\\\\\\\\\\\\\\\\\\\\\\\\\\\\\\\\\\\\\\\\\\\\\\\\\\\\\\\\ %%
%% PART 2.3 -- SETUP
%% //////////////////////////////////////////////////////////////////////// %%
\section{Setup}
\label{sec:dbsetup}
In this section, a new PostgreSQL user and database are created.  
The following steps are to be performed on either a native installation or within a virtual stack. 

Creating a new user allows for individualism in project development as well as added security since the user will have its own password. 
The new user should be given the privilege for creating databases (i.e., \texttt{CREATEDB}) and (optionally) the privilege to create other users (\texttt{CREATEUSER}). 
It is not necessary to create a new user. 

The new database is created based on the PostGIS template.  
BitNami LAPP stack installations require first that the PostGIS template be created.  
For native PostgreSQL and PostGIS installations, the PostGIS template should already be made available.  
In addition to the PostGIS extensions, the database utilizes pivot tables as a means of querying and retrieving data.  
To allow access to pivot tables, the tablefunc extension must be installed on the database. 
BitNami LAPP stack installations include the \texttt{tablefunc} extension.  
In the event that the \texttt{tablefunc} extension is not available, it is included in the additional facilities for postgreSQL package (e.g., \texttt{postgresql-contrib-9.1} under the Synaptic Package Manager in Linux).  
Available PostgreSQL extensions can be viewed by querying the pg\textunderscore 
available\textunderscore extensions table in the database.

\begin{enumerate}
    \item Create a new user (\emph{optional})
    \begin{enumerate}
        \item Log into the default PostgreSQL database:\\ 
              $>$~\texttt{psql -h 127.0.0.1 -p 5432 -U postgres -d postgres}
        \item On the PostgreSQL terminal line, create the new user account:\\ 
              $\#$~\texttt{CREATE USER user PASSWORD `*' CREATEDB CREATEUSER;}\\
              \\
              NOTE: Replace \texttt{user} with your name and 
              \texttt{`*'} with your own password.
        \item Quit PostgreSQL:\\ 
              $\#$~\texttt{\textbackslash q}
        \item Log into PostgreSQL as the newly created user:\\ 
              $>$~\texttt{psql -h 127.0.0.1 -p 5432 -U user -d postgres}
    \end{enumerate}
    \item Create a PostGIS template\\
          \emph{Note:} for native PostGIS installations (i.e., not a LAPP 
          stack installation) the template\textunderscore postgis is already 
          created
    \begin{enumerate}
        \item From the PostgreSQL terminal, create a new database:\\ 
              $\#$~\texttt{CREATE DATABASE template\textunderscore postgis;}
        \item Allow non superusers to create a database from this template:\\ 
              $\#$~\texttt{UPDATE pg\textunderscore database\\ 
              SET datistemplate=`true'\\ 
              WHERE datname=`template\textunderscore postgis';}\\
              \\
              Note: a response of ``UPDATE 1'' indicates success
        \item Load PostGIS SQL routines to the newly created database
        \begin{enumerate}
            \item Log out back into PostgreSQL to the newly created 
                  template\textunderscore postgis database as user postgres:\\ 
                  $>$~\texttt{psql -h 127.0.0.1 -p 5432 -U postgres \hfill \ldots \\
                  -d template\textunderscore postgis}
            \item Install PostGIS extensions to the template:\\ 
                  $\#$~\texttt{CREATE EXTENSION postgis;} \\
                  $\#$~\texttt{CREATE EXTENSION postgis\textunderscore topology;}
            \item Check that the extensions were installed correctly:\\ 
                  $\#$~\texttt{SELECT name, default\textunderscore version,
                  installed\textunderscore version\\ 
                  FROM pg\textunderscore available\textunderscore extensions\\ 
                  WHERE name LIKE `postgis\%';}\\
                  \\
                  Note: You should get a result showing postgis v. 
                  2.0.1 and postgis\textunderscore topology v. 2.0.1 or similar
            \item Enable users to alter the spatial tables:\\
                  $\#$~\texttt{GRANT ALL ON geometry\textunderscore columns TO 
                  PUBLIC;}\\
                  $\#$~\texttt{GRANT ALL ON geography\textunderscore columns TO 
                  PUBLIC;}\\
                  $\#$~\texttt{GRANT ALL ON spatial\textunderscore 
                  ref\textunderscore sys TO PUBLIC;}
            \item Log out of PostgreSQL:\\
                  $\#$~\texttt{\textbackslash q;}
            \item Check that everything is installed correctly:\\
                  $>$~\texttt{sudo -u postgres psql template\textunderscore 
                  postgis -c \hfill \ldots\\ 
                  ``SELECT 
                  PostGIS\textunderscore Full\textunderscore Version();''}
            \item Hit ``q'' on your keyboard to exit the results screen
        \end{enumerate}
    \end{enumerate}
    \item Create a new database based on PostGIS template:
    \begin{enumerate}
        \item Log into the default PostgreSQL database as user postgres
        \item From the PostgreSQL terminal, create a new database:\\
              $\#$~\texttt{CREATE DATABASE gepisat\\ 
              WITH OWNER=user\\ 
              TEMPLATE=template\textunderscore postgis;}\\
              \\
              Note: Use your user name for the owner or set owner to 
              `postgres' is you chose not to create a new user
        \item Log out of PostgreSQL
    \end{enumerate}
    \item Add extention to database to allow pivot tables:
    \begin{enumerate}
        \item Log into the newly created database:\\
              $>$~\texttt{psql -h 127.0.0.1 -p 5432 -d gepisat -U user}
        \item Add tablefunc extension:\\
              $\#$~\texttt{CREATE EXTENSION tablefunc;}
    \end{enumerate}
\end{enumerate}

An empty gepisat database is now ready for use.  
The next step is to create the three database tables and populate them with data.
