% gepisat_appdx_a.tex
%
% written by Tyler W. Davis
% Imperial College London
%
% 2014-10-29 -- created
% 2014-10-29 -- last updated
%
% ------------
% description:
% ------------
% This TEX file contains Appendix A: Python code snippets for the GePiSaT model documentation.
%
% ----------
% changelog:
% ----------
% 01. modularized chapter [14.10.29]
% 02. newline for each sentence [14.10.29]
% --> simpler for Git version control
%
%% \\\\\\\\\\\\\\\\\\\\\\\\\\\\\\\\\\\\\\\\\\\\\\\\\\\\\\\\\\\\\\\\\\\\\\\\ %%
%% APPENDIX A -- PYTHON CODE SNIPPETS
%% //////////////////////////////////////////////////////////////////////// %%
\section{Python Code Snippets:}

%% \\\\\\\\\\\\\\\\\\\\\\\\\\\\\\\\\\\\\\\\\\\\\\\\\\\\\\\\\\\\\\\\\\\\\\\\ %%
%% APPENDIX A.1 -- PIERCE_DEV.PY
%% //////////////////////////////////////////////////////////////////////// %%
\subsection{peirce\textunderscore dev.py}
\label{app:peircepy}
\texttt{ \\
\noindent 01~~import numpy\\
\noindent 02~~import scipy.special\\
\noindent 03~~\\
\noindent 04~~def peirce\textunderscore crit(N, n, m):\\
\noindent 05 \indent N = float(N)\\
\noindent 06 \indent n = float(n)\\
\noindent 07 \indent m = float(m)\\
\noindent 08 \indent if N > 1:\\
\noindent 09 \indent \indent Q = (n**(n/N)*(N-n)**((N-n)/N))/N\\
\noindent 10 \indent \indent Rnew = 1.0\\
\noindent 11 \indent \indent Rold = 0.0\\
\noindent 12 \indent \indent while ( abs(Rnew-Rold) > (N*2.0e-16) ):\\
\noindent 13 \indent \indent \indent ldiv = Rnew**n\\
\noindent 14 \indent \indent \indent if ldiv == 0:\\
\noindent 15 \indent \indent \indent \indent ldiv = 1.0e6\\
\noindent 16 \indent \indent \indent Lamda = ((Q**N)/(ldiv))**(1.0/(N-n))\\
\noindent 17 \indent \indent \indent x2 = 1.0 + (N-m-n)/n * (1.0-Lamda**2.0)\\
\noindent 18 \indent \indent \indent if x2 < 0:\\
\noindent 19 \indent \indent \indent \indent x2 = 0\\
\noindent 20 \indent \indent \indent \indent Rnew = Rold\\
\noindent 21 \indent \indent \indent else:\\
\noindent 22 \indent \indent \indent \indent Rold = Rnew\\
\noindent 23 \indent \indent \indent \indent Rnew = numpy.exp((x2-1)/2.0) \\ 
\noindent 24 \indent \indent \indent \indent \indent * scipy.special.erfc( \\  \noindent 25 \indent \indent \indent \indent \indent \indent numpy.sqrt(x2)/numpy.sqrt(2.0)\\
\noindent 26 \indent \indent \indent \indent \indent )\\
\noindent 27 \indent else:\\
\noindent 28 \indent \indent x2 = 0.0\\
\noindent 29 \indent return x2\\
}

\noindent Lines 1--2 import the necessary module libraries for performing the calculations.  
Lines 4--29 represent the function block for calculating Peirce's threshold error (i.e., $x^{2}$).  
Lines 5--7 cast the float data type to the input data.  
This is done to allow short hand (i.e., integer) values to be sent to the function.  
It is necessary for these values to be float-type to avoid integer division (i.e., whole number division). 
Line 8 is a check to make certain there is enough data for processing.  
Line 9 is the calculation of $Q$ (step 1).  
Line 10 is the initial guess for the value of $R$ (step 2).  
Line 11 initializes the old guess for the value of $R$ and is necessary to prompt the while loop (i.e., lines 12--26).  
Line 12 is the declaration of the while-loop criteria (i.e., convergence criteria for $R$). 
Line 13 calculates the denominator for a part of the $\lambda$ calculation.  
Line 14 checks the denominator to see if it is zero.  
If the denominator is zero, it is replaced with a 0.000001 to alleviate divide by zero issues (Line 15). 
Line 16 is the calculation of $\lambda$ (step 3).  
Line 17 is the calculation of $x^{2}$ (step 4).  
Line 18 checks to see if the value of $x^{2}$ has gone negative.  
If it has, its value is set to zero and $R$ is updated.  
Line 22 updates the previous guess for the value of $R$ before it is updated if $x^{2}$ is positive.  
Lines 23--26 calculate the new value of $R$ (step 5). 
Lines 27 and 28 set the value of $x^{2}$ to zero if there is not enough data for processing.  
When the convergence criteria is met for the while loop (i.e., line 12) the value for $x^{2}$ is returned.

\newpage

%% \\\\\\\\\\\\\\\\\\\\\\\\\\\\\\\\\\\\\\\\\\\\\\\\\\\\\\\\\\\\\\\\\\\\\\\\ %%
%% APPENDIX A.2 -- OUTLIER.PY
%% //////////////////////////////////////////////////////////////////////// %%
\subsection{outlier.py}
\label{app:outlierpy}
\texttt{ \\
\noindent 01~~import numpy\\
\noindent 02~~from scipy.optimize import curve\textunderscore fit\\
\noindent 03~~\\
\noindent 04~~(nee, ppfd) = monthly\textunderscore pairs()\\
\noindent 05~~if (len(ppfd) > 3 and len(nee) > 3):\\
\noindent 06 \indent (opt, cov) = curve\textunderscore fit(\\
\noindent 07 \indent \indent model\textunderscore h, ppfd, nee, estimates
\noindent 08 \indent \indent )\\
\noindent 09 \indent nee\textunderscore fit = model\textunderscore h(ppfd,
                     opt)\\
\noindent 10 \indent se = (nee - nee\textunderscore fit)**2.0\\
\noindent 11 \indent sse = sum(se)\\
\noindent 12 \indent mse = float(sse)/(len(nee) - 3.0)\\
\noindent 13 \indent x2 = peirce\textunderscore crit(len(nee), 1, 3)\\
\noindent 14 \indent d2 = mse*x2\\
\noindent 15 \indent n\textunderscore index = numpy.where(se > d2)[0]\\
\noindent 16 \indent n\textunderscore found = 
                     len(outliers\textunderscore index)\\
\noindent 17 \indent if n\textunderscore found == 0:\\
\noindent 18 \indent \indent x2 = peirce\textunderscore crit(len(nee), 2, 3)\\
\noindent 19 \indent \indent d2 = mse*x2\\
\noindent 20 \indent \indent n\textunderscore index = numpy.where(se > d2)[0]\\
\noindent 21 \indent \indent n\textunderscore found = 
                             len(outliers\textunderscore index)\\
\noindent 22 \indent n = 1\\
\noindent 23 \indent while (n <= n\textunderscore found):\\
\noindent 24 \indent \indent n += 1\\
\noindent 25 \indent \indent x2 = peirce\textunderscore crit(len(nee), n, 3)\\
\noindent 26 \indent \indent d2 = mse*x2\\
\noindent 27 \indent \indent n\textunderscore index = numpy.where(se > d2)[0]\\
\noindent 28 \indent \indent n\textunderscore found = 
                             len(outliers\textunderscore index)\\
\noindent 29 \indent ppfd\textunderscore ro = numpy.delete(ppfd, 
                     n\textunderscore index)\\
\noindent 30 \indent nee\textunderscore ro = numpy.delete(nee, 
                     n\textunderscore index)\\
}

\noindent Lines 1--2 load the necessary Python modules.  
Line 4 represents a function call to retrieve one month's NEE and PPFD pairs from a particular flux tower. 
Line 5 checks to make certain that enough data is available to perform the regression. 
Lines 6--8 calls the SciPy curve\textunderscore fit function which returns the optimization parameters for fitting the NEE and PPFD observations to model\textunderscore h (i.e., \ref{eq:hypmod}) given a set of initial parameter estimates.  
Line 9 takes the optimization parameters found by the curve\textunderscore fit function to get the model predictions of NEE (i.e., $u$ in equation \ref{eq:residerr}).  
Lines 10--12 calculate the squared-error (SE), the sum of the squared-error (SSE), and the mean squared-error (MSE).  
Line 13 calculates Peirce's deviation ($x^{2}$) while Line 14 calculates the threshold squared error ($\Delta^{2}$).  
Lines 15--16 identify and count the instances of where model squared-error exceeds the threshold deviation.  
Lines 17--21 performs a secondary check for outliers in the case where no are initially found.  
Lines 22--28 perform an iterative search for additional outliers by incrementing Peirce's $n$ parameter until the number of outliers identified is less than the number of outliers assumed (i.e., Line 23).  
Lines 29--30 remove the outliers from the original datasets. 

\newpage

%% \\\\\\\\\\\\\\\\\\\\\\\\\\\\\\\\\\\\\\\\\\\\\\\\\\\\\\\\\\\\\\\\\\\\\\\\ %%
%% APPENDIX A.3 -- NETCDF.PY
%% //////////////////////////////////////////////////////////////////////// %%
\subsection{netcdf.py}
\label{app:netcdfpy}
\texttt{ \\
\noindent 01~~import numpy\\
\noindent 02~~from scipy.io import netcdf\\
\noindent 03~~\\
\noindent 04~~doc = path + filename\\
\noindent 05~~(thisyear, thismonth) = get\textunderscore month(doc)\\
\noindent 06~~f = netcdf.NetCDFFile(doc, "r")\\
\noindent 07~~voi = `SWdown'\\
\noindent 08~~sh\textunderscore day, sh\textunderscore lat, 
              sh\textunderscore lon = f.variables[voi].shape\\
\noindent 09~~for y in xrange(sh\textunderscore lat):\\
\noindent 10 \indent pxl\textunderscore lat = f.variables[`lat'].data[y]\\
\noindent 11 \indent for x in xrange(sh\textunderscore lon):\\
\noindent 12 \indent \indent stationid = 720*y + x\\
\noindent 13 \indent \indent for t in xrange(sh\textunderscore day):\\
\noindent 14 \indent \indent \indent thisday = t+1\\
\noindent 15 \indent \indent \indent timestamp = datetime.date(\\
\noindent 16 \indent \indent \indent \indent thisyear, thismonth, thisday\\
\noindent 17 \indent \indent \indent \indent )\\
\noindent 18 \indent \indent \indent pxl\textunderscore val = 
                                     f.variables[voi].data[t,y,x]\\
\noindent 19 \indent \indent \indent if pxl\textunderscore val < 1.0e6 and 
                                     pxl\textunderscore lat > -60:\\
\noindent 20 \indent \indent \indent \indent data = process\textunderscore 
                                             watch(voi, pxl\textunderscore val, 
                                             timestamp)\\
\noindent 21~~f.close()\\
}

\noindent Lines 1--2 load the necessary Python modules.  
Line 4 defines the netCDF file in terms of its path and file name.  
Line 5 represents an assignment of the current file's associated year and month.  
In most cases, the year and month can be read directly from the file name; however, there are other alternatives to this.  
Line 6 opens the netCDFfile for reading.  
Line 7 defines the variable of interest, in this case it is shortwave downwelling solar radiation ($SW_{down}$).  
Line 8 saves the shape of $SW_{down}$ in terms of the number of days and the number of pixels along the latitude and longitude.  
Line 9 begins iterating through the latitude pixels while Line 10 saves the current latitude value (in decimal degrees).  
Line 11 begins the iteration through the longitude.  
Line 12 calculates the $stationid$ parameter (as defined in section \ref{sec:dbmtdes}). 
Line 13 starts the iteration through the days while Line 14  saves the current month's day.  
Lines 15--17 create a \texttt{datetime.date} object based on the three fields which make up the current day.  
Line 18 reads the $SW_{down}$ value for the given day, latitude, and longitude.  
Line 19 filters erroneous values (i.e., $SW_{down} \geq 10^{6}$) and observations from Antarctica (i.e., $lat \leq -60^{\circ}$).  
Line 20 represents the data processing of the valid $SW_{down}$ observations.  
Line 21 closes the netCDF file.

\newpage

%% \\\\\\\\\\\\\\\\\\\\\\\\\\\\\\\\\\\\\\\\\\\\\\\\\\\\\\\\\\\\\\\\\\\\\\\\ %%
%% APPENDIX A.4 -- HDF.PY
%% //////////////////////////////////////////////////////////////////////// %%
\subsection{hdf.py}
\label{app:hdfpy}
\texttt{ \\
\noindent 01~~import numpy\\
\noindent 02~~from pyhdf import SD\\
\noindent 03~~\\
\noindent 04~~doc = path + filename\\
\noindent 05~~ts = get\textunderscore modis\textunderscore ts(doc)\\
\noindent 06~~f = SD.SD(doc)\\
\noindent 07~~voi = ''CMG 0.05 Deg Monthly EVI''\\
\noindent 08~~f\textunderscore select = f.select(voi)\\
\noindent 09~~f\textunderscore data = f\textunderscore select.get()\\
\noindent 10~~f.end()\\
\noindent 11~~(sh\textunderscore lat, sh\textunderscore lon) = data.shape\\
\noindent 12~~for y in xrange(sh\textunderscore lat):\\
\noindent 13 \indent for x in xrange(sh\textunderscore lon):\\
\noindent 14 \indent \indent zval = data[y][x]\\
\noindent 15 \indent \indent evi = zval / 10000.0\\
}

Lines 1--2 load the necessary Python modules.  
Note that \texttt{pyhdf} is not a simple installation for native Python environments.  
It is recommended to use a third-party Python developing and programming environment (see Section \ref{sec:mespy}). 
Line 4 defines the HDF file in terms of its path and filename.  
Line 5 represents the assignment of a timestamp (i.e., a \texttt{datetime} object) based on the HDF filename.  
Line 6 opens the HDF file for reading.  
Line 7 defines the variable of interest, in this case it is ``CMG 0.5 Deg Monthly EVI'' corresponding to MODIS 0.5$^{\circ}$ degree resolution EVI monthly product.  
Line 8 selects the variable of interest from the HDF file.  
Line 9 retrieves the associated data (as a numpy array).  
Line 10 closes the HDF file.  
Line 11 saves the array shape of the data (in terms of the number of pixels along latitude and longitude). 
Lines 12--13 iterate through each pixel (in terms of x and y coordinates).  
Line 14 reads the pixel value associated with the x-y coordinate (EVI $\times$ 10,000).  
Line 15 converts the pixel value to EVI (ranges from -0.2--1.0).
